\section{Sicherheit}


\subsection{\question{Nenne einige absichtliche und unabsichtliche Angriffe auf Hardware, Software und/oder Daten.}}
\begin{answer}
Wie können solche Angriffe klassifiziert werden?
absichtliche Angiffe:
- Hardware:
Zerstörung, herbeiführen von Störungen, Diebstahl
- Software:
Zerstörung, Fälschung, Kopie, Viren, Würmer
unabsichtliche Angiffe:
- Hardware:
Speisen, Getränke, Wasserschäden, Mäuse
- Software:
Bugs, Bedienfehler
Klassifizierung:
- Abfangen: Anzapfen, Manipulation
- Modifikation: Daten verfälschen, Programme verändern
- Unterbrechung: Zerstörung, Löschen von Daten

\end{answer}

\subsection{\question{Welche grundsätzlichen Sicherheitsziele kann man unterscheiden? Was ist eine Sicherheitspolitik?}}
\begin{answer}
Eine Sicherheitspolitik soll das reibungslose Erfüllen der vom System zu erledigenden Aufgaben
sicherstellen. Dabei befinden sich die einzelnen Faktoren in einem Spannungsfeld, hohe Kosten
stehen z.B. immer wieder einer besonders hohen Sicherheit gegenüber. Entsprechend muss im
Einzelfall die Wahrscheinlichkeit einer Betriebsstörung und ihre möglichen Folgen gegen den Sicherheitsaufwand
abgewogen werden.
Als grundsätzliche Sicherheitsziele können Gemeinhaltung, Unversehrtheit und Verfügbarkeit unterschieden
werden.
\end{answer}

\subsection{\question{Auf welche verschiedenen Arten kann sich ein Benutzer authentifizieren?}}
\begin{answer}
Wissen: z.B. Passwort
Besitz: z. B. Schlüssel, Chipkarte
Persönliche Merkmale: z.B. Fingerabdruck, Iris, Sprache
\end{answer}

\subsection{\question{Welche Komponenten enthält eine Zugriffskontrollmatrix? Wie ordnen sich die Dateizugriffsrechte in UNIX in dieses Schema ein?}}
\begin{answer}
Eine dreidimensionale Zugriffskontrollmatrix enthält für jeden User, jede Datei Lese- und Schreibrechte.
Die drei Dimensionen sind also User-Identety, Rechte und Datei.
Die Dateizugriffsrechte in UNIX ordnen dem Besitzer einer Datei, der zugehörigen Gruppe und
dem "Rest der Welt" Lese-, Schreib- und Ausführungsrechte zu.
\end{answer}

\subsection{\question{Charakterisiere symetrische und asymetrische Verschlüsselungsverfahren. Wie können sie zur Realiesierung einer Vertraulichkeit eingesetzt werden? Warum werden häufig Mischformen eingesetzt?}}
\begin{answer}
Bei symetrischer Verschlüsselung existieren zwei gleiche Schlüssel. Das hat den Nachteil, das bei bekannt werden des Schlüssels die Vertraulichkeit nicht mehr gegeben ist.
Bei asymetrischer Verschlüsselung existieren insgesamt vier Schlüssel: zwei private und zwei öffentliche Schlüssel. Die öffentlichen Schlüssel können allgemein bekannt sein. Die Verschlüsselung selbst passiert mit dem privaten Schlüssel des Senders und dem öffentlichen Schlüssel des Empfängers.
Die Entschlüsselung einer Nachricht erfolgt mit dem privaten Schlüssel des Empfängers und dem
öffentlichen Schlüssel des Senders.

Asymetrische Verschlüsselung ist erheblicher rechenintensiver als Symetrische. Allerdings ist die
symetrische Verschlüsselung unsicherer als die Asymetrische. Deshalb werden oft Mischformen
eingesetzt, z.B. werden die symetrischen Schlüssel mit Hilfe asymetrischer Verschlüsselung ausgetauscht
um die Schlüsselübergabe so sicher als möglich zu machen, die Kommunikation selbst
findet dann aber mit Hilfe der symetrischen Verschl üsselung statt um eine angemessene Geschwindigkeit
(z.B. Videokonferenzen) zu gewährleisten.
\end{answer}